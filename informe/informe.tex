\documentclass[a4paper]{article}

\usepackage[spanish]{babel}
\usepackage[utf8]{inputenc}
\usepackage{amsmath}
\usepackage{graphicx}
\usepackage[colorinlistoftodos]{todonotes}
\usepackage{hyperref}
\usepackage{multicol}
\usepackage{makeidx}
\usepackage{hyperref}
\usepackage{caption}
\usepackage{amsfonts}
\usepackage{amssymb}
\usepackage[utf8]{inputenc}
\usepackage{verbatim}
\usepackage{listings}
\usepackage{float}
\lstset{language=C++, showstringspaces=false, tabsize=2, breaklines=true, title=\lstname}

\usepackage[margin=0.5in]{geometry}

\title{TP1 Tleng}

\author{-}

\date{\today}

\makeindex

\begin{document}
\newgeometry{margin=2cm}
\pagenumbering{gobble}
\raggedleft
\includegraphics[width=8cm]{logo_dc.jpg}\\

\raggedright
\vspace{3cm}
{\Huge \bfseries Trabajo Práctico 1}
\rule{\textwidth}{0.02in}
\large Jueves 2 de julio de 2015 \hfill Teoría de Lenguajes
\vspace{1.5cm}

 
\centering \LARGE 
\vspace{1.5cm}

\normalsize
\begin{tabular}{|l@{\hspace{4ex}}c@{\hspace{4ex}}l|}
        \hline
        \rule{0pt}{1.2em}Integrante & LU & Correo electr\'onico\\[0.2em]
        \hline
        \rule{0pt}{1.2em} Aleman, Damián Eliel &377/10 &\tt damian\_8591@hotmail.com \\[0.2em]
        \rule{0pt}{1.2em} Gauna, Claudio Andrés &733/06 &\tt gauna\_claudio@yahoo.com.ar\\[0.2em]        
        \rule{0pt}{1.2em} Vargas Telles, Matías & 318/12 &\tt mvt208@hotmail.com \\[0.2em]
        \hline
\end{tabular}

\vspace{1.0cm}
\raggedright

\begin{multicols}{2}
\includegraphics[width=8cm]{logo_uba.jpg}

\columnbreak
\vspace*{4.5cm}
\raggedleft
\textbf{Facultad de Ciencias Exactas y Naturales}\\
Universidad de Buenos Aires\\
\small
Ciudad Universitaria - (Pabellon I/Planta Baja)\\
Intendente G\"uiraldes 2160 - C1428EGA\\
Ciudad Autonoma de Buenos Aires - Rep. Argentina\\
Tel/Fax: (54 11) 4576-3359\\
http://www.fcen.uba.ar
\end{multicols}

\restoregeometry

\clearpage

\pagenumbering{arabic}

\tableofcontents

\vspace{3cm}

\clearpage


\section{Introducción}
Para el presente trabajo se nos solicitó la implementación de diversos algoritmos para autómatas finitos determinísticos.
\linebreak

Este informe tiene como objetivo acompañar el código entregado para la resolución de los enunciados planteados, permitiendo así su mejor comprensión.\linebreak



En las siguientes secciones describiremos la representación elegida para los autómata y luego agregaremos documentación sobre los algoritmos implementados.


\section{Clase automaton}
Representamos un automaton con una clase de java. La misma contiene (como atributos privados) un conjunto de caracteres $\sigma$, un diccionario de transiciones \textit{transitions}, un arreglo de estados \textit{states}, un estado inicial \textit{initialState} y un conjunto de estados finales \textit{finalStates}.
\linebreak
La interfaz pública permite:
\begin{itemize}
\item Crear un automaton
\item Obtener el iésimo estado
\item Obtener las transiciones
\item Consultar si un estado es final
\item Preguntar por una transición particular (desde un estado mediante un caracter)
\item Completa un automaton con transiciones lambda.
\item Obtener los estados (iniciales y finales) y saber si un string es aceptado.
\end{itemize}


\subsection{Clase Automaton Operations}
La interfaz pública permite :
\begin{itemize}
\item Crear el complemento de otro automata
\item Crear el automata minimizado
\item Crear la intersección de dos automatas
\item Crear la unión de dos automatas
\item Crear la concatenación de dos automatas
\item Computa si dos automatas son equivalentes
\item Determinizar un automata
\end{itemize}

\subsection{Clase Automaton reader}
En esta clase leemos un archivo que representa un automata con el formato del enunciado y devolvemos un automaton.


\subsection{Clase Automaton writer}
En esta clase escribimos el automata que tenemos como entrada y lo escribimos en un archivo.
También se encarga de crear el archivo .dot para graficar automatas.


\subsection{Clase Regular expression reader}
En esta clase leemos un archivo que representa una expresión regular con el formato del enunciado y devolvemos un automaton.

\subsection{Clase State }
Mediante esta clase representamos al estado, que se crea con un nombre

\subsection{Clase Transition }
Mediante esta clase representamos una transición entre estados. Podemos calcular el origen de la transición, así como también su destino y símbolo correspondiente.


\section{Algoritmos principales implementados}

En esta sección describiremos brevemente los algoritmos empleados.
\subsection{Minimización de AFD}
Este algoritmo minimiza el automata clasificando a los estados en clases de estados indistinguibles. Se dice que dos estados son indistinguibles cuando para toda transición con origen en los dos estados tiene el mismo destino(o llegan a estados que son indistinguibles, es decir que pertenecen a la misma clase de equivalencia).

\subsection{Decidir si una palabra pertenece al lenguaje}
Esto lo implementamos en la función reconoce (en la clase Automaton).
Básicamente, el algoritmo comienza con el estado inicial y va recorriendo desde allí mediante las transiciones, consumiendo los símbolos que forman la palabra. Si se consumieron todos los símbolos y se logró llegar a un estado final, la palabra pertenece al lenguaje, en caso contrario no pertenece al lenguaje.  

\subsection{Determinización de AFND}
Para un mejor siguemiento de este algoritmo recomendamos tener a mano su pseudocódigo\footnote{ \url{http://web.cecs.pdx.edu/~harry/compilers/slides/LexicalPart3.pdf}}.
Observamos que en este algoritmo la explosión combinatoria puede generar autómatas con una cantidad de estados notoriamente mayor que los originales, lo cual prolonga considerablemente los tiempos de ejecución y el uso de memoria. Por ejemplo:

\begin{figure}[H]
\centering
\includegraphics[scale=0.5]{detExample.jpg}
\caption{AFND original con 3 estados}
\end{figure}


\begin{figure}[H]
\centering
\includegraphics[scale=0.25]{detExample2.jpg}
\caption{AFD resultante con 26 estados}
\end{figure}

\subsection{Clausura $\lambda$}
Para cada nuevo estado de la determinización es necesario realizar la clausura $\lambda$, es decir, obtener todos los estados alcanzables via transiciones vacías. Para ello procedimos con un algoritmo estilo BFS (con una cola) obteniendo los estados "vecinos" por transiciones $\lambda$.
\section{Conclusiones}
En la realización de este trabajo nos topamos con las dificultades que representan la implementación y la ejecución de los algoritmos vistos en clase para AFND (sobre todo aquellos de tiempo exponencial, como la determinización).  

\end{document}
